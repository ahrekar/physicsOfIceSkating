\documentclass[12pt]{article}
\usepackage{amsmath,amsthm,amssymb}
\usepackage{hyperref}

\title{Physics of Ice Skating: The Joy Of Physics}
\author{Karl Ahrendsen \and Andrew Vikartofsky}
\date{\today}

\begin{document}
\maketitle
\section{Slipperiness}
\subsection{Intro.}
Hi, my name is Karl ( Andrew does his thing ) and we're both graduate
students in our third year of Physics studies at UNL. 
We're going to start our talk tonight with a question we'd like you to consider:
Why is ice slippery? Now that I've given you a couple seconds to think about that, 
could I get a brave volunteer to share what came to your mind? Anyone? I knew it would
be a bad idea to start the talk with a rhetorical question, I'm just going to start 
over from the beginning. (pause, turn and face away from audience for 3 seconds, 
turn back around, smiling) Fat penguin. Sorry, I just wanted to break the ice. 
(* Comment on the effectiveness of jokes as introduction tools *)
    
Seriously
though, any volunteers for why ice is slippery? Andrew and I asked ourselves
this same question when we were preparing for the talk, and these
are some of the answers that we came up with. 

(* Advance to slipperiness Slide *)

We thought smoother things
would be more slippery. This seemed to in general be true, but brought up
another interesting question. Though we can feel something and make some
guess as to how slippery it is, what if the roughness is small enough that we 
can't detect it with our sense of touch? It seems plausible that we could make a 
material that feels smooth, or rough, but on a smaller scale has the opposite 
behavior. 

We also thought that things being wet would in general always make them more 
slippery, until we thought of a counter-example, a dry sponge slides easily
across a table, but a wet sponge will quickly come to a halt.

We then came up with an idea that neither of us were sure was correct,
That things which are cold are naturally more slippery. We decided we had
to investigate. 

This is what scientists do in general, we ask questions and come up with a 
plausible answer. Then, we get to test whether that answer is correct. 

\subsection{Poking the Universe}
We're going to do some science right now 
and test: if something is colder, is it more slippery? When we want to test 
if something is slippery, we put it on an angled surface so that it just barely 
starts moving. The smaller the angle, the more slippery the object is. The 
larger the angle, the less slippery. 
What we've done here is fixed the angle so that a metal block will just \emph{barely}
start moving if we place it on the ramp. 

Now we've brought with us some 
liquid nitrogen for purposes of quickly cooling objects. 
At this point, I have to note that Liquid Nitrogen is 
dangerous, and should only be handled with the proper protective gear. 

If we cool down our metal block
we can see if the slipperiness changes. 

Remember now, we set up our ramp so that the metal would start moving if
we give it just the tiniest push. Let's say it becomes more slippery, what 
would happen then? Good. Let's do it and see what happens. 
We can also try it with another material, wood, and see if the results are 
consistent.

So we find out that coldness does not indicate slipperiness.
We, of course, could have arrived at this solution simply by observing that
concrete doesn't become more slippery when it's 90 degrees out versus 20 degrees. 

This is at the heart of what physicist, and in general scientists, do. We ask 
a question, come up with a possible answer, and then test whether that answer is
correct. As an experimental physicist, I actually get to poke the universe and
see what happens, while Andrew, a theorist, does the same thing but there isn't actually 
physical poking, he writes computer code to simulate the poking. 

So now that
we've done some experimental physics, let's turn to the theorist for some
explanation of why ice is slippery. 

\subsection{The real answer}

(* HAND MIC TO ANDREW AS HE DROPPEST THE KNOWLEDGE *)


\section{Practical Knowledge}
The practical knowledge I want to impart to you is how you can
use physics concepts to avoid falling down on the 
ice.

\subsection{Center of Mass}
The key to not falling is keeping 
our center of mass above our balance points. 

(* Show Rocks COM slide *)

We have here three examples of objects fulfilling that requirement, 
and therefore balancing. 

Let's break this down a bit. The balance point is just where the object is
supported. Center of mass is a technical term that means where, on average
your weight is located. In these images, the center of the weight must 
be directly above wherever the balance point is. 

Our balance points
are our feet, or more accurately our ice skates. If we have both
feet on the ground we also get all the area in between our feet.
Imagine that you have a single rubberband wrapped around both feet.
The space inside that rubber band is where you have to keep your
center of mass in order to balance. Therefore, you have a larger balance
area, if you have your feet slightly apart rather than close
together. 

The center of mass for a person standing
straight up it's located approximately  below the belly button and 
halfway between our front and back. 

(* Show Center of Mass Slide *)

But if you then bend your body, the center of mass changes. If I lean forward, 
my center of mass shifts forward as the top of my body moves forward,
you can conveniently tell when the center of mass goes outside your
balance area, because you have to take a step to avoid 
falling over. 

In general, the lower your center
of mass, the easier it is to balance. So that means that someone
like me (very tall), is going to have a more difficult time ice skating than
someone shorter. It also means that \emph{anyone} can increase their 
balance by bending their knees a bit, lowering their center of mass. 
This has the added benefit of bringing you closer to the ground so 
a fall doesn't hurt as much.

Lowering your center of mass won't help with the major 
difficulty 
of balancing on ice, that is that your feet don't want to stay 
where you put them! Thankfully, the ice skate's design helps 
with this problem. 

(* Advance to blank slide *)
\subsection{Controlling Slippery/ Skate Design}

I have here an ice skate that the John Breslow Ice Hockey 
center was kind enough to let us borrow.

When we skate across the ice, the metal blade
points in the direction of our motion and allows us to easily travel. 
Here, I am moving the skate across the surface of a synthetic ice 
material. 

But! If it allows us to easily travel forward, the same has to also be true 
for backward, which leaves us with a problem. 
How can we move forward if when we try to push forward our skate slips 
on the ice? The answer
is that we must use the sides of the blade to push off. 

(* Advance to skaters diagram *)

The long
edge of the skate does not slide as easily across the ice because the
skates ``bite" into the ice. So here's the key:
If you start to lose your feet from under you,
you should push \emph{out} rather than moving your feet forward and 
backward to 
give yourself stability. Let's look a bit more into this ``biting" that
the ice skates do on the ice. 

If you look at the bottom of a skate, you might be surprised to see that it looks relatively
flat, rather than being sharpened to a point. If you look still closer, you'll see that 
though it looks flat from afar, it actually is concave, a word which here means that the
edges of the blade go down further than the center of the blade, giving in effect, two small 
edges that are skated upon.

(* Advance to slide with ice skate hollows *)

These edges dig in to the ice and when the skates are pushed from side
to side (rather than forward and backward), they prevent the skate from slipping in that direction. 
If this concavity were absent, ice skates would be just about as
effective at skating on ice as tennis shoes- which is about as effective as a hog on ice.

(* Advance to pig slipping all over ice *)

When the skates are completely flat, 
You lose the ability of the skate to dig in
and they just slide every which-way across the ice. 

Hopefully this will be practical knowledge for the next time you're on 
the ice (hopefully tonight). Our last topic deals with ideas 
of a much more fundamental nature, so I'll pass it back to our 
theorist, for which the fundamental is actually very practical.

\section{Linear Motion}
\subsection{Intro}
\subsection{Linear Momentum}
\section{Circular Motion/Angular motion}

\section{Conclusion}
I couldn't think of a good conclusion, so I wrote a rap
about the differing perspectives of why ice is slippery 
through history. It's inspired by a popular musical that
also happens to be about history.

I'm (K Ahrendsen -- Vikartofsky) and it's plain to see 
the water on the ice- it's coming from T.
That's the temperature, the pressure makes it rise. 
Bring in that heat and you'll be starting to slide. 

Wee wee on to me, I think there's something you missed.
Pressure causes rise in temp., sure, but consider dis.
The rise is small, so that can't be all. 
There must be something else that's causing all these people to fall. 

Blah! Blah! I know! It's friction! That's the cause of it!
Take two things, rub 'em together, like a pair-o-sticks.
Sticks- you get fire, now don't call me a liar,
If you do this on the ice its not gonna be drier!

Wow! Those ideas are great, but I gotta new uhn.
The liquids already there- don' gotta do nothin'.
The air sets molecules free to move and warm things up
We did experiments and all, so don't even try to fuuuuihnd
...another reason that the ice is slippery.
\end{document}
