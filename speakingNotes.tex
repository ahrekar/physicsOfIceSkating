\documentclass[12pt]{article}
\usepackage{amsmath,amsthm,amssymb}
\usepackage{hyperref}

\title{Physics of Ice Skating: The Joy Of Physics}
\author{Karl Ahrendsen}
\date{\today}

\begin{document}
\maketitle
\section{Slipperiness}
\subsection{Intro.}
Hi, my name is Karl ( Andrew does his thing ) and we're both graduate
students in our third year of Physics studies at UNL. 
We're going to start our talk tonight with a question we'd like you to consider:
Why is ice slippery? Now that I've given you a couple seconds to think about that, 
could I get a brave volunteer to share what came to your mind? Anyone? I knew it would
be a bad idea to start the talk with a rhetorical question, I'm just going to start 
over from the beginning. (pause, turn and face away from audience for 3 seconds, 
turn back around, smiling) Fat penguin. Sorry, I just wanted to break the ice. 
    
Seriously
though, any volunteers for why ice is slippery? Andrew and I asked ourselves
this same question when we were preparing for the talk, and I came up with the
idea that ice is slippery because it is cold. 
This is what scientists do in general, we ask questions and come up with a 
plausicle answer. Then, we get to test whether that answer is correct. 

\subsection{Poking the Universe}
We're going to do some science right now 
and test: if something is colder, is it more slippery? When we want to test 
if something is slippery, we put it on an angled surface so that it just barely 
starts moving. The smaller the angle, the more slippery the object is. The 
larger the angle, the less slippery. 
What we've done here is fixed the angle so that a metal block will just \emph{barely}
start moving if we place it on the ramp. 

Now We've brought with us some 
liquid nitrogen for purposes of quickly cooling objects. 
At this point, I have to note that Liquid Nitrogen is 
dangerous, and should only be handled with the proper protective gear. 

Now if we cool down our metal block
we can see if the slipperiness changes. Let's say it becomes more slippery, what 
would happen then? Good. Let's do it and see what happens. 
We can also try it with another material, wood, and see if the results are 
consistent.

So we find out that coldness does not indicate slipperiness.
We, of course, could have arrived at this solution simply by observing that
concrete doesn't become more slippery when it's 90 degrees out versus 20 degrees. 

This is at the heart of what physicist, and in general scientists, do. We ask 
a question, come up with a possible answer, and then test whether that answer is
correct. As an experimental physicist, I actually get to poke the universe and
see what happens, while Andrew, a theorist, does the same thing but there isn't actually 
physical poking, he writes computer code to simulate the poking. 

So now that
we've done some experimental physics, let's turn to the theorist for some
explanation.

\subsection{The real answer}
So what is the real answer though? Why \emph{is} ice is slippery? 
It turns out that this is a more difficult question than you would think, and
it wasn't until the year 2000 that we obtained experimental evidence for our 
current best explanation of Ice's slidiness. We've known for a long time
that water on top of ice plays a large role in its slipperiness. Over the years
the explanation of how that water gets there has been debated. 

(* HAND MIC TO ANDREW AS HE DROPPEST THE KNOWLEDGE *)

The best explanation that scientists have so far is that even below freezing 
temperatures (as low as $-35^o$C ($-31^{o}$F)) the surface of ice is always 
covered with a very thin layer of water. While we're down here talking about 
-35 degree temperatures, I can't help but mention a fun fact. -40 degrees is a 
very special temperature because you don't have to indicate whether its 
Fahrenheit or Celsius, they're both the same temperature. 

When you step on the ice,  this quasi-liquid layer is created, and the
water molecules 
act like little roller-balls underneath your feet, causing you to slip and fall. 
If you're thinking like a physicist now, you'll ask yourself, "Wait, would having
little roller-balls underneath my feet really be \emph{that} slippery." A 
comprehensive search of the literature (in this case, when I say literature, 
what I actually mean is YouTube) provided no experimental 
evidence for this claim, but we were able to find a close approximation to 
the rollerballs that gives this claim some credibility (* video of Home-Alone
car slipping scene. *) 

In addition, the scientists that experimentally discovered why ice was slippery 
also discovered that if ice didn't have this thin-liquid layer, it would be 
just as slippery as wet concrete, which 
isn't that slippery!

\section{Practical Knowledge}
Not only is physics great for explaining why ice is slippery, 
it also gives us some pratical knowledge for how we can keep
ourselves from falling down. 

\subsection{Center of Mass}
The key to avoiding a bruised behind is keeping 
our center of mass above our balance points. Our balance points
are our feet, or more accurately our ice skates. And if we have both
feet on the ground we also get all the area in between our feet.
Imagine that you have a single rubberband wrapped around both feet.
The space inside that rubber band is where you have to keep your
center of mass.

Center of mass is a technical term that just means where, on average
your weight is located.  For humans if you're standing
straight up it's located approximately  below our belly button. 
If you then bend your body, the center of mass changes. If I lean forward, 
my center of mass shifts with my body, and if I get far enough so my center 
of mass is outside my balance area, I have to take a step to avoid 
falling over. 

The lower your center
of mass, the easier it is to balance. So that means that someone
like me, is going to have a more difficult time ice skating than
someone shorter. It also means that anyone can increase their 
balance by bending their knees a bit, lowering their center of mass. 
This has the added benefit of bringing you closer to the ground so 
a fall doesn't hurt as much.

Lowering your center of mass won't help with the major difficulty 
with balancing on ice though, your feet don't want to stay 
where you put them! Thankfully, the ice skate's design helps 
with this problem. 


\subsection{Controlling Slippery/ Skate Design}
When we skate across the ice, the metal blade
points in the direction of our motion and allows us to easily travel. 
If it allows us to easily travel forward, the same has to also be true 
for backward, which leaves us with a problem. 
How can we move forward if when we try to push off our skate slips 
on the ice? The answer
is that we must use the sides of the blade to push off. The long
edge of the skate does not slide as easily across the ice because the
skates ``bite" into the ice. If you start to lose your feet from under you,
push \emph{out} rather than moving your feet forward and backward to 
give yourself stability. Let's look a bit more into this ``biting" that
the ice skates do on the ice. 

If you look at the bottom of a skate, you might be surprised to see that it looks relatively
flat, rather than being sharpened to a point. If you look still closer, you'll see that 
though it looks flat from afar, it actually is concave, a word which here means that the
edges of the blade go down further than the center of the blade, giving in effect, two small 
edges that are skated upon. These edges dig in to the ice when they are pushed from side
to side (rather than forward and backward) and prevent the skate from slipping in that direction. 
If this concavity were absent, ice skates would be just about as
effective as skating on ice as tennis shoes- which is about as effective as a hog on ice.
(show video of pig slipping all over ice). When the surface is completely flat, 
You lose the ability of the skate to dig in
and they just slide every which-way across the ice. 

\section{Linear Motion}
\subsection{Intro}
One of a physicists favorite things is conservation 
laws. Energy is always conserved. Momentum is also always conserved. 

We're going to do something now, and we're a little bit nervous about it.
We didn't want to have any equations in our talk, because people often 
don't get along with equations. But we thought things would work best 
if we brought in a couple simple equations. 

(* NASTY MATH SLIDE *)

Okay, so this is really simple, so just hang with me here, alright? 
This F here is something called the ``Field Tensor," and 
contained within it is all of the electric and magnetic fields in space... 
\subsection{Linear Momentum}
Momentum is something
that people are often familiar with, but in physics it has a very specfic definition-
an object's mass multiplied by its velocity.
Here's a beautiful example of momentum being conserved. If we have two people standing next to each other on 
an ice rink one is a football player, the other a cheerleader. The two people aren't moving,
so each of their velocities is zero. If the velocities are zero, then the momentum is 
also zero, and the total momentum of the system is- you guessed it, zero. 
What do you think happens if they push off of one another? 

Because we know that the total momentum is conserved, we know that it will still be zero.
But we also know that they'll move away from each other! How can the total momentum be
zero if they're both moving? Direction matters! One is a positive value, and the other
is negative, and their numbers exactly match each other. Negative numbers to the 
rescue!
But the cheerleader will move much more quickly
than the football player because he or she has a smaller mass. In fact, the veolcity 
of the cheerleader will be inversely proportional to his or her own mass, and directly 
proportional to the mass of the football player and the football player's velocity. 

There's a side point that I'd like to make here.
As you can already tell, describing how the world behaves with words starts to get 
really cumbersome. It's no secret that we physicists love equations, and now you
can hopefully understand why. I find that many times people say they enjoy learning
about physics, but don't like the math and equation part of it. If I don't use
math and equations, then to explain what happens I have to say this:
(show slide with all the words that I used to 
describe the problem earlier)
But if I use equations, I can just say this:
(Show slide with momentum equation). 
Equations are wonderful because they communicate a wealth of information in a very 
small space. 

Conservation laws are one of the foundations of physics, and they also help to explain
one of the most interesting phenomena in Ice skating: how the people can manage to 
spin around in circles so quickly. 

\section{Circular Motion/Angular motion}
(* Show video of ice skater spinning very quickly *)
You see, what is true about things moving in a straight line- momentum being conserved. Is also 
true about things that are rotating about a point. We call it conservation of angular momentum,
and it has a couple of key differences from the linear case. Just like linear momentum, 
angular momentum depends on the velocity of the object, this time the angular velocity,
or how fast it is spinning. Now instead of multiplying the velocity by mass, it's
multiplied by something called a moment of inertia. The moment of inertia is something
that you usually use calculus to calculate, so we're not going to talk about it too much.
What's important for this discussion is that the moment of inertia depends on the shape
of the thing that's spinning. If more stuff is further away from the center of the spin, 
then it will have a large moment of inertia. If the same stuff is closer to the center
of the spin, it will have a smaller moment of inertia. We multiply the moment of inertia
by the angular velocity and we angular momentum, which is always conserved. 

We have an excellent way to show
this here. This is called a Haberman's sphere. It's a cool toy that can expand or contract
to many times its original size. We just learned about moment of inertia, if we were to spin 
this on the string its hanging on while its expanded, does it have a small, or large moment
of inertia? Yes, Large. And then we pull the string, what happens to the moment of inertia? 
Correct again, it gets smaller. Let's see conservation of momentum in action: 
(* Do demo of haberman's sphere *)

So now we start the sphere turning while its expanded, and since we know it has a moment of 
inertia, and now an angular velocity, the angular momentum, L, also has a value that's 
fixed in size. If I were then to pull the string, we change the moment of inertia, but since 
angular momentum is \emph{conserved} the L value has to stay the same size, that means that
what has to happen to the angular velocity? Right, I gets smaller, but L has to be the same
size, so it spins faster.

Since neither Andrew or I are trained Ice Skating professionals, we can't demo this for you 
on ice, but we have a close approximation with this rotating disc. (Depending on time, 
either one of us can do it or we can call up a volunteer). You begin a spin with your arms
extended out away from you. Once the spin has begun, you bring your arms closer 
to your body, the spinning center. This reduces your moment of inertia, but your total
angular momentum has to say the same, so you start spinning faster.

\section{Conclusion}
I couldn't think of a good conclusion, so I wrote a rap
about the differing perspectives of why ice is slippery 
through history. It's inspired by a popular musical that
also happens to be about history.

I'm (K Ahrendsen -- Vikartofsky) and it's plain to see 
the water on the ice- it's coming from T.
That's the temperature, the pressure makes it rise. 
Bring in that heat and you'll be starting to slide. 

Wee wee on to me, I think there's something you missed.
Pressure causes rise in temp., sure, but consider dis.
The rise is small, so that can't be all. 
There must be something else that's causing all these people to fall. 

Blah! Blah! I know! It's friction! That's the cause of it!
Take two things, rub 'em together, like a pair-o-sticks.
Sticks- you get fire, now don't call me a liar,
If you do this on the ice its not gonna be drier!

Wow! Those ideas are great, but I gotta new uhn.
The liquids already there- don' gotta do nothin'.
The air sets molecules free to move and warm things up
We did experiments and all, so don't even try to fuuuuihnd
...another reason that the ice is slippery.
\end{document}
